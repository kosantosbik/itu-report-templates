% Copyright (c) 2016 Ongun Kanat <ongun.kanat@gmail.com>
% This document is a free software licensed under MIT license.
% For redistribution details look at COPYING file.

% 12pt and ISO A4 paper with title page add notitlepage for otherwise
\documentclass[a4paper, 12pt, titlepage]{article}

% 2cm margin from all sides
\usepackage[a4paper,margin=2cm]{geometry}

% Use American English for dates etc.
\usepackage[american]{babel}
% If document is in Turkish then use
% \usepackage[turkish]{babel}
% or for both
% \usepackage[turkish,american]{babel}

% Indent at section beginnings
% \usepackage{indentfirst} % look at below for reverse
% Paragraph spacings set parindent to 0
\setlength{\parindent}{0pt}
\setlength{\parskip}{12pt}

% utf-8 support
\usepackage[utf8]{inputenc}

% Graphics for PDFTeX
\usepackage[pdftex]{graphicx}

% Figure placement
\usepackage{float}

% An enumeration package for flexible enumeration
\usepackage{enumitem}

% Courier monospace font
\usepackage{courier}

% Links, both local and external
\usepackage{hyperref}
\hypersetup{
	unicode=true,
	colorlinks=true,
	urlcolor=blue,
	citecolor=black,
	menucolor=black,
	linkcolor=black
}

% Figure captions are bold
\usepackage[labelfont=bf]{caption}

% Pseudocode
\usepackage{algorithmicx}
\usepackage{algpseudocode}
\usepackage{algorithm}

% Syntax highlighting simple
\usepackage{listings}
\lstset{basicstyle=\ttfamily,frame=lines,tabsize=4}
\renewcommand{\lstlistingname}{Code}

% Syntax higlighting (advanced)
%\usepackage{minted}

% Title, author and date info
\title{My Glorious Report}
\author{Besim Ongun Kanat \\ 150120000}
\date{December 22, 2016}

\begin{document}
% Fix Turkish fix hypenation
%\shorthandoff{=}

% For a generic title page one can use standard \maketitle command
% It will use the title info above
% \maketitle

% The title page can be made by hand as below
\begin{titlepage}
	\begin{center}
		\large{Istanbul Technical University \\ Faculty of Computer and Informatics \\ Computer Engineering Department} \\
		\vspace{150pt}
		\Large{BLG 100E \\ The Glorious \LaTeX \\ Report}  \\
		\vspace{30pt}
		\Large{\textbf{Group MadCoders}} \\
		\large{Besim Ongun Kanat - 150120047} \\
		\vspace{\fill} % Fill out until the page end
		\large{December 2\textsuperscript{nd}, 2016}
	\end{center}
\end{titlepage}
\pagenumbering{roman}
\newpage
\tableofcontents
\newpage

% For the ones who doesn't know: 1,2,..9 called West Arabic numbers
\pagenumbering{arabic}
\section{Introduction}
I made a \LaTeX template to help my friends on creating good looking reports.

\section{General stuff}
\subsection{Text styles}
You can make text \textbf{bold}, \textit{italic}, \underline{underlined} or in \texttt{typewriter fonts}. You can \textit{\textbf{use}} \texttt{them \textbf{combined}}

\begin{center}
	You can make paragraphs centered
\end{center}

\begin{flushright}
	Or right aligned
\end{flushright}

\paragraph{The important paragraph:}
We can create titled paragraphs

\subsection{Enumeration and lists}
With the help of \emph{enumitem} package we can create numbered lists as below:

\begin{enumerate}[label=\arabic*]
	\item Apples (We can use nested lists)
	\begin{enumerate}[label=\Alph*)]
		\item Starking
		\item Golden
	\end{enumerate}
	\item Kiwis
	\item and of course Bananas!
\end{enumerate}

We can also create unordered lists
\begin{itemize}
	\item Ford Prefect
	\item Arthur Dent
	\item Zaphod Beeblebrox
\end{itemize}

\subsection{Tables}

Creating tables can become a bit annoying the [H] here ensures the table is displayed where it is defined.

\begin{table}[H]
	\label{table:music}
	\caption{Table of great music}
	\centering
	\begin{tabular}{c | r | l |}
		left column center aligned & center column right aligned & right column left aligned \\ % don't forget to end the line
		\hline
		\hline % Any number of horizontal lines
		\hline
		I want to break free & We're the champions & Bohemian Rhapsody \\
		\hline
	\end{tabular}
\end{table}

% As wikibooks suggests we can draw more complex tables

\begin{table}[H]
	\begin{flushright}
		\begin{tabular}{|r|l|}
			\hline
			7C0 & hexadecimal \\
			3700 & octal \\ \cline{2-2}
			11111000000 & binary \\
			\hline \hline
			1984 & decimal \\
			\hline
		\end{tabular}
	\end{flushright}
\end{table}

For more info consult \href{https://en.wikibooks.org/wiki/LaTeX/Tables}{Wikibooks}.

We can continue here but... \newpage
sometimes a clear page in our life is much better.

\vspace{150pt}

{\LARGE This page is left blank intentionally} \\


\newpage

\section{Images and Figures}

We can include images like:

\begin{figure}[H]
	\centering
	\caption{GTA}
	\label{fig:gta}
	\includegraphics[width=120pt]{gta.jpg}
\end{figure}

scale them relative to page width

\begin{figure}[H]
	\centering
	\caption{GTA2}
	\label{fig:gta2}
	\includegraphics[width=.65\textwidth]{gta.jpg} % scale 75%
\end{figure}

or even we can include PDFs (tip: Save SVG images as PDF) and they can scale (try to zoom in, it will not get pixelated)

\begin{figure}[H]
	\centering
	\caption{Linux's mascot: Tux}
	\label{fig:tux}
	\includegraphics[width=.30\textwidth]{tux.pdf} % scale 75%
\end{figure}

\newpage

\section{Inserting Code Pieces}
\subsection{Pseudocode}
\begin{algorithm}[H]
	\caption{The depth first search algorithm}
	\label{algo:dfs}
	\begin{algorithmic}
	\State \textbf{Graph} $G$
	\State \textbf{Node} $start$
	\Function{Depth-First-Search}{$G$, $start$}
		\State \textbf{Tree} $ T $ \Comment The resulting search tree
		\State \textbf{Stack} $ S $ \Comment An empty stack
		\State \textbf{Set} $ V $ \Comment An empty set of visited nodes
		\State \Call{set-root}{$ T $,$ current $}
		\State \Call{push}{$ S $,$ start $}
		\While{\Call{not-empty}{S}}
			\State $current \gets$ \Call{pop}{$ S $}
			\If{\textbf{not} \Call{Contains}{$ V $, $ current $} }
				\State \Call{insert}{$ V $, $ current $}
				\ForAll{$ n $ : \Call{neighbors}{$ current $} }
					\State \Call{push}{$ S $, $ n $}
					\State \Call{insert-sub-node}{$ T $, $ current $, $ n $}
					\Comment Insert node to subtree of $ current $
				\EndFor
			\EndIf
		\EndWhile
		\State \Return $ T $
	\EndFunction
	\end{algorithmic}
\end{algorithm}

\subsection{Real Code}

\begin{lstlisting}[language=C++,caption=Depth first search in C++]
class Graph
{
	set<int> nodes;
	vector< vector<int> > edge_list;
public:
	void dfs();
}

\end{lstlisting}

\end{document}
